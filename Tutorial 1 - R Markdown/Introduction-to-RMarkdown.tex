% Options for packages loaded elsewhere
\PassOptionsToPackage{unicode}{hyperref}
\PassOptionsToPackage{hyphens}{url}
%
\documentclass[
]{article}
\usepackage{amsmath,amssymb}
\usepackage{lmodern}
\usepackage{ifxetex,ifluatex}
\ifnum 0\ifxetex 1\fi\ifluatex 1\fi=0 % if pdftex
  \usepackage[T1]{fontenc}
  \usepackage[utf8]{inputenc}
  \usepackage{textcomp} % provide euro and other symbols
\else % if luatex or xetex
  \usepackage{unicode-math}
  \defaultfontfeatures{Scale=MatchLowercase}
  \defaultfontfeatures[\rmfamily]{Ligatures=TeX,Scale=1}
\fi
% Use upquote if available, for straight quotes in verbatim environments
\IfFileExists{upquote.sty}{\usepackage{upquote}}{}
\IfFileExists{microtype.sty}{% use microtype if available
  \usepackage[]{microtype}
  \UseMicrotypeSet[protrusion]{basicmath} % disable protrusion for tt fonts
}{}
\makeatletter
\@ifundefined{KOMAClassName}{% if non-KOMA class
  \IfFileExists{parskip.sty}{%
    \usepackage{parskip}
  }{% else
    \setlength{\parindent}{0pt}
    \setlength{\parskip}{6pt plus 2pt minus 1pt}}
}{% if KOMA class
  \KOMAoptions{parskip=half}}
\makeatother
\usepackage{xcolor}
\IfFileExists{xurl.sty}{\usepackage{xurl}}{} % add URL line breaks if available
\IfFileExists{bookmark.sty}{\usepackage{bookmark}}{\usepackage{hyperref}}
\hypersetup{
  pdftitle={Markdown Tutorial},
  pdfauthor={Arsh Kumbhat; Vibha Rao; Aditi Singh; Jainil Dharmil Shah; Revendranath T},
  hidelinks,
  pdfcreator={LaTeX via pandoc}}
\urlstyle{same} % disable monospaced font for URLs
\usepackage[margin=1in]{geometry}
\usepackage{graphicx}
\makeatletter
\def\maxwidth{\ifdim\Gin@nat@width>\linewidth\linewidth\else\Gin@nat@width\fi}
\def\maxheight{\ifdim\Gin@nat@height>\textheight\textheight\else\Gin@nat@height\fi}
\makeatother
% Scale images if necessary, so that they will not overflow the page
% margins by default, and it is still possible to overwrite the defaults
% using explicit options in \includegraphics[width, height, ...]{}
\setkeys{Gin}{width=\maxwidth,height=\maxheight,keepaspectratio}
% Set default figure placement to htbp
\makeatletter
\def\fps@figure{htbp}
\makeatother
\setlength{\emergencystretch}{3em} % prevent overfull lines
\providecommand{\tightlist}{%
  \setlength{\itemsep}{0pt}\setlength{\parskip}{0pt}}
\setcounter{secnumdepth}{-\maxdimen} % remove section numbering
\ifluatex
  \usepackage{selnolig}  % disable illegal ligatures
\fi

\title{Markdown Tutorial}
\author{Arsh Kumbhat \and Vibha Rao \and Aditi Singh \and Jainil Dharmil
Shah \and Revendranath T}
\date{9/20/2021}

\begin{document}
\maketitle

\hypertarget{r-markdown}{%
\subsection{R Markdown}\label{r-markdown}}

This is an R Markdown document. Markdown is a simple formatting syntax
for authoring HTML, PDF, and MS Word documents. For more details on
using R Markdown see \url{http://rmarkdown.rstudio.com}.

When you click the \textbf{Knit} button a document will be generated
that includes both content as well as the output of any embedded R code
chunks within the document. You can embed an R code chunk like this:

\hypertarget{what-is-r-markdown}{%
\subsection{What is R Markdown?}\label{what-is-r-markdown}}

Markdown is a lightweight markup language that you can use to add
formatting elements to plaintext text documents.

\hypertarget{why-use-markdown}{%
\subsection{Why Use Markdown?}\label{why-use-markdown}}

\begin{itemize}
\tightlist
\item
  Markdown can be used for everything. People use it to create websites,
  documents, notes, books, presentations, email messages, and technical
  documentation.
\item
  Markdown is portable.
\item
  Markdown is platform independent. You can create Markdown-formatted
  text on any device running any operating system.
\item
  Markdown is everywhere. Websites like Reddit and GitHub support
  Markdown, and lots of desktop and web-based applications support it.
\end{itemize}

R Markdown is an open-source tool for producing reproducible reports in
R. It enables you to keep all of your code, results, plots, and writing
in one place. R Markdown provides an authoring framework for data
science. You can use a single R Markdown file to both :-

\begin{enumerate}
\def\labelenumi{\arabic{enumi}.}
\tightlist
\item
  save and execute code.
\item
  generate high quality reports that can be shared with an audience.
\end{enumerate}

R Markdown documents are fully reproducible and support dozens of static
and dynamic output formats.

R Markdown is particularly useful when you are producing a document for
an audience that is interested in the results from your analysis, but
not your code.

With R Markdown, you have the option to export your work to numerous
numerous formats including PDF, Microsoft Word, a slideshow, or an HTML
document for use in a website.

We'll use the RStudio integrated development environment (IDE) to
produce our R Markdown reference guide.

\hypertarget{workflow}{%
\subsection{Workflow:}\label{workflow}}

\begin{enumerate}
\def\labelenumi{\arabic{enumi}.}
\tightlist
\item
  Open a new .Rmd File
\item
  Write/ Edit your document - add tables, figures, citations, etc.
\item
  Knit document to create report - set output formats and options in the
  YAML header
\item
  Save and Render
\item
  Share your work(if interested)
\end{enumerate}

\hypertarget{installing-r-markdown}{%
\subsubsection{Installing R Markdown:}\label{installing-r-markdown}}

\texttt{install.packages("rmarkdown")}

\hypertarget{install-latex-tinytex-for-pdf-reports}{%
\subsubsection{Install LaTeX (TinyTeX) for PDF
reports:}\label{install-latex-tinytex-for-pdf-reports}}

\texttt{tinytex::install\_tinytex()}

\hypertarget{to-uninstall-tinytex-run}{%
\subsubsection{To uninstall TinyTeX,
run}\label{to-uninstall-tinytex-run}}

\texttt{tinytex::uninstall\_tinytex()}

When R Markdown is converted to PDF, Pandoc converts Markdown to an
intermediate LaTeX document first. The R package tinytex has provided
helper functions to compile LaTeX documents to PDF.

\hypertarget{code-chunks-and-inline-r-code-}{%
\subsection{Code Chunks and Inline R
Code:-}\label{code-chunks-and-inline-r-code-}}

A code chunk usually starts with ```\{\} and ends with ```. You can
write any number of lines of code in it. Inline R code is embedded in
the narratives of the document using the syntax \texttt{r}

\hypertarget{compile-an-r-markdown-document-}{%
\subsection{Compile an R Markdown
Document:-}\label{compile-an-r-markdown-document-}}

\begin{itemize}
\tightlist
\item
  The usual way to compile an R Markdown document is to click the Knit
  button as shown in Figure 2.1, and the corresponding keyboard shortcut
  is Ctrl + Shift + K (Cmd + Shift + K on macOS).
\item
  RStudio calls the function rmarkdown::render() to render the document
  in a new R session.
\item
  If you want to render a document in the current R session, you can
  also call rmarkdown::render() by yourself, and pass the path of the
  Rmd file to this function.
\item
  The second argument of this function is the output format, which
  defaults to the first output format you specify in the YAML metadata
  (if it is missing, the default is html\_document).
\item
  When you have multiple output formats in the metadata, and do not want
  to use the first one, you can specify the one you want in the second
  argument, e.g., for an Rmd document foo.Rmd with the metadata:
\end{itemize}

\hypertarget{you-can-render-it-to-pdf-via}{%
\subsubsection{You can render it to PDF
via:}\label{you-can-render-it-to-pdf-via}}

\texttt{\{rmarkdown::render(\textquotesingle{}foo.Rmd\textquotesingle{},\ \textquotesingle{}pdf\_document\textquotesingle{})\}}

\hypertarget{output-formats}{%
\subsubsection{Output Formats:}\label{output-formats}}

Output formats in R Markdown can be broadly classified into two
categories:\\
1.Documents 2.Presentations

\hypertarget{some-common-types-of-documents-that-can-be-created-using-r-markdown}{%
\subsubsection{Some common types of Documents that can be created using
R
Markdown:}\label{some-common-types-of-documents-that-can-be-created-using-r-markdown}}

\begin{enumerate}
\def\labelenumi{\arabic{enumi}.}
\tightlist
\item
  word\_document
\item
  pdf\_document
\item
  latex\_document
\item
  html\_document
\item
  github\_document
\end{enumerate}

\hypertarget{some-common-types-of-presentations-that-can-be-created-using-r-markdown}{%
\subsubsection{Some common types of Presentations that can be created
using R
Markdown:}\label{some-common-types-of-presentations-that-can-be-created-using-r-markdown}}

\begin{enumerate}
\def\labelenumi{\arabic{enumi}.}
\tightlist
\item
  powerpoint\_presentation
\item
  beamer\_presentation
\item
  ioslides\_presentation
\item
  slidy\_presentation
\end{enumerate}

\hypertarget{common-markdown-features}{%
\subsection{Common Markdown Features:}\label{common-markdown-features}}

\begin{enumerate}
\def\labelenumi{\arabic{enumi}.}
\tightlist
\item
  Text Formats
\item
  Lists
\item
  Hyperlinks
\item
  Mathematical Expressions
\item
  Images and Tables with Captions
\item
  Templates
\item
  Speakers' Notes(Especially when creating presentations)
\end{enumerate}

\hypertarget{document-elements}{%
\subsection{Document Elements:}\label{document-elements}}

\hypertarget{a.-insert-page-breaks}{%
\subsubsection{A.) Insert page breaks:}\label{a.-insert-page-breaks}}

\begin{itemize}
\tightlist
\item
  When you want to break a page, you can insert the command
  \texttt{\{\textbackslash{}newpage\}} in the document.
\item
  It is a LaTeX command, but the rmarkdown package is able to recognize
  it for both LaTeX output formats and a few non-LaTeX output formats
  including HTML,5 Word, and ODT.
\end{itemize}

\textbf{Note} : For HTML output, page breaks only make sense when you
print the HTML page, otherwise you will not see the page breaks, because
an HTML page is just a single continuous page.

\hypertarget{b.-combine-words-into-a-comma-separated-phrase}{%
\subsubsection{B.) Combine words into a comma separated
phrase:}\label{b.-combine-words-into-a-comma-separated-phrase}}

\begin{itemize}
\tightlist
\item
  The function \texttt{\{knitr::combine\_words()\}} can be used to
  concatenate words into a phrase regardless of the length of the
  character vector.
\item
  Basically, for a single word, it will just return this word; for two
  words A and B, it returns ``A and B''; for three or more words, it
  returns ``A, B, C, \ldots, Y, and Z''.
\item
  The function also has a few arguments that can customize the output.
\end{itemize}

\hypertarget{examples}{%
\subsubsection{Examples:}\label{examples}}

v \textless- c(``apple'', ``banana'', ``cherry'')
knitr::combine\_words(v)\\
\emph{output} : apple, banana, and cherry

knitr::combine\_words(v, before = ```'', after = "`")\\
\emph{output}: 'apple', `banana', and `cherry'

knitr::combine\_words(v, and = "")\\
\emph{output}: apple, banana, cherry

knitr::combine\_words(v, sep = " / ``, and =''")\\
\emph{output}: apple / banana / cherry

knitr::combine\_words(v{[}1{]}) \# a single word\\
\emph{output}: apple

knitr::combine\_words(v{[}1:2{]}) \# two words\\
\emph{output}: apple and banana

knitr::combine\_words(LETTERS{[}1:5{]})\\
\emph{output}: A, B, C, D, and E

\hypertarget{formatting}{%
\subsection{Formatting:}\label{formatting}}

We will look at a small task of creating interesting presentations using
R Markdown which would give us an overall review of the various
formatting and other advanced features offered by R Markdown.

\hypertarget{assignmentstake-home-task}{%
\subsection{Assignments/Take Home
Task:}\label{assignmentstake-home-task}}

Create a PDF LaTex Beamer Presentation OR HTML Ioslides Presentation of
about 7 - 10 slides in R Markdown. Please note that you are advised not
to create a PowerPoint Presentation since that is already being
discussed in the class.

Thank You!

\hypertarget{references}{%
\subsection{References:}\label{references}}

\begin{enumerate}
\def\labelenumi{\arabic{enumi}.}
\tightlist
\item
  \url{https://bookdown.org/yihui/rmarkdown-cookbook}
\item
  \url{https://bookdown.org/yihui/rmarkdown/}
\item
  R Markdown Cheatsheets (shared in the Google Drive Folder)
\end{enumerate}

\end{document}
